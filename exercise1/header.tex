%
% MMP Ex Sheet Template
%
% Rafael Kueng <rafik@physik.uzh.ch>
% originally based on the template from the Jetzter group
%

%
% USAGE:
% create a new tex file for each new exercise. use like this:
%
% \input{./_header}
% \input{./_shortcuts}
% 
% \newcommand{\sheetnr}{1}
% \newcommand{\issued}{16.09.2014}
% \newcommand{\dueUni}{22.09.2014 9:00}
% 
% \startsheet
% 
% \exercise[Nr Pts]{Descripntion}
% \begin{subtasks}
%   \task blablabla blasdf asdf dgsf
%   \task sadfkj sdf asdf
% \end{subtasks}
% \turnpage

%%%%%%%%%%%%%%%%%%%%%%%%%%%%%%%%%%%%%%%%%%%%%%%%%%%%%%%%%%%%%%%%%%%%%%%%%%%%%%%%%


%
% Imports
%

\documentclass[11pt,a4paper]{article}
% debug
%\documentclass[11pt,a4paper,draft]{article}
%\usepackage{showframe}

\usepackage[utf8]{inputenc}
\usepackage[english]{babel}
\usepackage[T1]{fontenc}
\usepackage{lmodern}

\usepackage{graphicx}
\usepackage{amsmath}
\usepackage{amsfonts}
\usepackage{mathtools}
\usepackage{siunitx} % help with si units, like: \si{\kilogram\metre\per\second} and \num{4.458757e-12}
\usepackage{upgreek} % upright greek letters, like: $\uppi$
\usepackage{xparse}

%\usepackage{psfrag}
%\usepackage{enumerate}
\usepackage{enumitem}
\usepackage{hyperref}
\usepackage{geometry}
\usepackage{xifthen}
\usepackage{verbatim}  % used for the \comment used to hide solutions
\usepackage{xcolor}
\usepackage{framed}
\usepackage{cancel}
\usepackage{ulem} % for different underlines http://texdoc.net/texmf-dist/doc/generic/ulem/ulem.pdf
\usepackage{braket}

%
% Setup the layout
%

\linespread{1.15}
\geometry{left=2cm,right=2cm,top=2cm,bottom=3cm}

\setlength{\parindent}{0em}
\setlength{\parskip}{0.5em}

\pagestyle{empty}


%
% Set the authors ect..
%

\newcommand{\thelecture}{PHY211 - KTI}
\newcommand{\thelectureshort}{FP}
\newcommand{\thelectureURL}{\url{http://www.physik.uzh.ch/de/lehre/PHY211/}}
\newcommand{\sheettype}{Exercise}
\newcommand{\thesemester}{Herbstsemester 2016}
\newcommand{\thelecturer}{Prof.~N.~Serra}
\newcommand{\theassistants}{R.~Silva~Coutinho, P.~Owen}



%
% DEFINITIONS & NEW COMMANDS
%

\newcounter{exercise}
\setcounter{exercise}{1}

% define switches
\newboolean{turnpagesw}
\setboolean{turnpagesw}{true}

\newboolean{showsol}
\setboolean{showsol}{false}



% starts the doc and draws the header / banner
\newcommand{\startsheet}{
  \begin{document}
  \begin{minipage}{0.25\textwidth}{
    \begin{flushleft}\includegraphics[width=2.5cm]{./unisiegel.pdf}\end{flushleft}
  }\end{minipage}
  \hfill
  \begin{minipage}{0.40\textwidth}{
    \begin{center}\bf\LARGE%
    \textrm{\thelecture}\\
    \LARGE\textrm{\sheettype \, Sheet \sheetnr}\end{center}
  }\end{minipage}
  \hfill
  \begin{minipage}{0.25\textwidth}{
    \begin{flushright}
      \textrm{\thesemester}\\
      \textrm{\thelecturer} \\
    \end{flushright}
  }\end{minipage}

  \begin{minipage}{0.6\textwidth}{
    \theassistants\\
    \thelectureURL
  }\end{minipage}
  \hfill
  \begin{minipage}{0.3\textwidth}{
    \begin{flushright}
      Issued: \issued \ifthenelse{\isundefined{\version}}{}{ {\tiny(v\version)}}\\
      Due: \dueUni
    \end{flushright}
  }\end{minipage}

  \rule{\textwidth}{1pt}
}

% use this to break a page
% creates a turn over note and creates the headline on the new page
\newcommand{\turnpage}{
    \ifthenelse{\boolean{turnpagesw}}{
        \vfill\hfill -- please turn over -- \newpage
        {Exercises for \thelectureshort} \hfill Sheet \sheetnr\vspace{-0.2cm}\\
        \rule{\textwidth}{1pt}
    }{}}

% use this to start a new exercise
\newcommand{\exercise}[2][?]{
  \vspace{1.5em}
  \textbf{Exercise \theexercise:} #2 (#1 Pts.) %
  \stepcounter{exercise}
}

% use this to create subtaks:
\newenvironment{subtasks}{
  \newcommand{\task}{\item }
  \setlength{\itemsep}{1pt}
  \setlength{\parskip}{0pt}
  \setlength{\parsep}{0pt}
  \begin{enumerate}[label=\alph*)]
}
{
  \end{enumerate}
}



% set this "flag" at beginning if you want to print the solutions as well
\newcommand{\showsolutions}{
	\setboolean{showsol}{true}
	\setboolean{turnpagesw}{false} % don't create pageturns if solutions are shown
}


% mark solutions with this envirionment
\newenvironment{solution}{
  \ifthenelse{\boolean{showsol}}
    {
    \begin{framed}
    \textbf{SOLUTION:}
    }
    {\expandafter\comment}
}
{
    \ifthenelse{\boolean{showsol}}
    {
    \end{framed}
    \newpage
    }
    {\expandafter\endcomment}
}


