% vim: set sts=4 et:

%
% MMP Ex Sheet Template
%
% Rafael Kueng <rafik@physik.uzh.ch>
% originally based on the template from the Jetzter group
%

%
% USAGE:
% create a new tex file for each new exercise. use like this:
%
% \input{./_header}
% \input{./_shortcuts}
% 
% \newcommand{\sheetnr}{1}
% \newcommand{\issued}{16.09.2014}
% \newcommand{\dueUni}{22.09.2014 9:00}
% 
% \startsheet
% 
% \exercise[Nr Pts]{Descripntion}
% \begin{subtasks}
%   \task blablabla blasdf asdf dgsf
%   \task sadfkj sdf asdf
% \end{subtasks}
% \turnpage

%%%%%%%%%%%%%%%%%%%%%%%%%%%%%%%%%%%%%%%%%%%%%%%%%%%%%%%%%%%%%%%%%%%%%%%%%%%%%%%%%


%
% Imports
%

\documentclass[11pt,a4paper]{article}
% debug
%\documentclass[11pt,a4paper,draft]{article}
%\usepackage{showframe}

\usepackage[utf8]{inputenc}
\usepackage[english]{babel}
\usepackage[T1]{fontenc}
\usepackage{lmodern}

\usepackage{graphicx}
\usepackage{amsmath}
\usepackage{amsfonts}
\usepackage{mathtools}
\usepackage{siunitx} % help with si units, like: \si{\kilogram\metre\per\second} and \num{4.458757e-12}
\usepackage{upgreek} % upright greek letters, like: $\uppi$
\usepackage{xparse}

%\usepackage{psfrag}
%\usepackage{enumerate}
\usepackage{enumitem}
\usepackage{hyperref}
\usepackage{geometry}
\usepackage{xifthen}
\usepackage{verbatim}  % used for the \comment used to hide solutions
\usepackage{xcolor}
\usepackage{framed}
\usepackage{cancel}
\usepackage{ulem} % for different underlines http://texdoc.net/texmf-dist/doc/generic/ulem/ulem.pdf
\usepackage{braket}

%
% Setup the layout
%

\linespread{1.15}
\geometry{left=2cm,right=2cm,top=2cm,bottom=3cm}

\setlength{\parindent}{0em}
\setlength{\parskip}{0.5em}

\pagestyle{empty}


%
% Set the authors ect..
%

\newcommand{\thelecture}{PHY211 - KTI}
\newcommand{\thelectureshort}{FP}
\newcommand{\thelectureURL}{\url{http://www.physik.uzh.ch/de/lehre/PHY211/}}
\newcommand{\sheettype}{Exercise}
\newcommand{\thesemester}{Herbstsemester 2016}
\newcommand{\thelecturer}{Prof.~N.~Serra}
\newcommand{\theassistants}{R.~Silva~Coutinho, P.~Owen}



%
% DEFINITIONS & NEW COMMANDS
%

\newcounter{exercise}
\setcounter{exercise}{1}

% define switches
\newboolean{turnpagesw}
\setboolean{turnpagesw}{true}

\newboolean{showsol}
\setboolean{showsol}{false}



% starts the doc and draws the header / banner
\newcommand{\startsheet}{
  \begin{document}
  \begin{minipage}{0.25\textwidth}{
    \begin{flushleft}\includegraphics[width=2.5cm]{./unisiegel.pdf}\end{flushleft}
  }\end{minipage}
  \hfill
  \begin{minipage}{0.40\textwidth}{
    \begin{center}\bf\LARGE%
    \textrm{\thelecture}\\
    \LARGE\textrm{\sheettype \, Sheet \sheetnr}\end{center}
  }\end{minipage}
  \hfill
  \begin{minipage}{0.25\textwidth}{
    \begin{flushright}
      \textrm{\thesemester}\\
      \textrm{\thelecturer} \\
    \end{flushright}
  }\end{minipage}

  \begin{minipage}{0.6\textwidth}{
    \theassistants\\
    \thelectureURL
  }\end{minipage}
  \hfill
  \begin{minipage}{0.3\textwidth}{
    \begin{flushright}
      Issued: \issued \ifthenelse{\isundefined{\version}}{}{ {\tiny(v\version)}}\\
      Due: \dueUni
    \end{flushright}
  }\end{minipage}

  \rule{\textwidth}{1pt}
}

% use this to break a page
% creates a turn over note and creates the headline on the new page
\newcommand{\turnpage}{
    \ifthenelse{\boolean{turnpagesw}}{
        \vfill\hfill -- please turn over -- \newpage
        {Exercises for \thelectureshort} \hfill Sheet \sheetnr\vspace{-0.2cm}\\
        \rule{\textwidth}{1pt}
    }{}}

% use this to start a new exercise
\newcommand{\exercise}[2][?]{
  \vspace{1.5em}
  \textbf{Exercise \theexercise:} #2 (#1 Pts.) %
  \stepcounter{exercise}
}

% use this to create subtaks:
\newenvironment{subtasks}{
  \newcommand{\task}{\item }
  \setlength{\itemsep}{1pt}
  \setlength{\parskip}{0pt}
  \setlength{\parsep}{0pt}
  \begin{enumerate}[label=\alph*)]
}
{
  \end{enumerate}
}



% set this "flag" at beginning if you want to print the solutions as well
\newcommand{\showsolutions}{
	\setboolean{showsol}{true}
	\setboolean{turnpagesw}{false} % don't create pageturns if solutions are shown
}


% mark solutions with this envirionment
\newenvironment{solution}{
  \ifthenelse{\boolean{showsol}}
    {
    \begin{framed}
    \textbf{SOLUTION:}
    }
    {\expandafter\comment}
}
{
    \ifthenelse{\boolean{showsol}}
    {
    \end{framed}
    \newpage
    }
    {\expandafter\endcomment}
}



%
% Define here some handy shortcuts
% 
% to be consistent and backwards compatible, try not to delete
% or replace existing ones
%


% \renewcommand{\d}{\mathrm{d}}
% \renewcommand{\labelenumi}{(\roman{enumi})}

% \newcommand{\bm}{\boldmath}
% \newcommand{\dps}{\displaystyle}
% \newcommand{\e}{\mbox{e}}
% \newcommand{\del}{\partial}
% \newcommand{\arctanh}{\mathrm{arctanh}}
% \newcommand{\prim}[1]{{#1^{\prime}}}
% \newcommand{\tr}{\mathrm{Tr}}
% \newcommand{\naeher}{\!\!\!}
% \newcommand{\vect}[1]{\vec #1}
% \newcommand{\ddd}[1]{\d\vect{#1} }

\newcommand{\IN}{\mathbb{N}}
\newcommand{\INO}{\mathbb{N}_0}
\newcommand{\IR}{\mathbb{R}}
\newcommand{\IC}{\mathbb{C}}

\renewcommand{\Re}{\operatorname{Re}}
\renewcommand{\Im}{\operatorname{Im}}

\DeclareMathOperator{\arsinh}{arsinh}
\DeclareMathOperator{\arcosh}{arcosh}
\DeclareMathOperator{\artanh}{artanh}
\DeclareMathOperator{\arcoth}{arcoth}

% commonly misstyped...
\DeclareMathOperator{\arcsinh}{arsinh}
\DeclareMathOperator{\arccosh}{arcosh}
\DeclareMathOperator{\arctanh}{artanh}
\DeclareMathOperator{\arccoth}{arcoth}

\DeclareMathOperator*{\Res}{Res}  % residuum
\DeclareMathOperator{\Ang}{Ang}  % Angle

\newcommand{\PI}{\uppi}    % consult http://tex.stackexchange.com/questions/54166/upright-lowercase-pi
\newcommand{\I}{\textrm{i}} %\upiota} % http://tex.stackexchange.com/questions/223759/how-to-add-a-dot-above-imath
\newcommand{\E}{\mathrm{e}}

\DeclarePairedDelimiter\abs{\lvert}{\rvert}

\newcommand{\Order}[1]{\mathcal{O}\!\left(#1\right)}
\newcommand{\D}{\mathrm{d}}
\newcommand{\pderr}[2]{\frac{\partial #1}{\partial #2}}
\newcommand{\integ}[2]{\int \D #2 \; \left[#1\right]}

% sofisticated integral.. 
% \Int[lower][upper]{int_variable}{term}
% \Int*[lower][upper]{int_variable}{term} --> without brackets around term
\NewDocumentCommand{\Int}{ s O{} O{} m m }{
	\IfBooleanTF {#1}
	{\int_{#2}^{#3} \D #4 \; #5}
	{\int_{#2}^{#3} \D #4 \; \left[#5\right]}
}

% handy derrivative
% \DD[power]{variable}
\NewDocumentCommand{\DD}{ O{x} O{} }{
	\frac{\D^{#2}}{\D#1^{#2}}
}
\NewDocumentCommand{\DDx}{ O{} }{\DD[x][#1]}
\NewDocumentCommand{\DDy}{ O{} }{\DD[x][#1]}
\NewDocumentCommand{\DDt}{ O{} }{\DD[x][#1]}


\newcommand{\half}{\frac{1}{2}}
\newcommand{\oneover}[1]{\frac{1}{#1}}

\newcommand{\textoversymb}[2]{\mathrel{\overset{\makebox[0pt]{\mbox{\normalfont\tiny\sffamily #2}}}{#1}}}


%\DeclarePairedDelimiter\parents{\left(}{\right)}
%\DeclarePairedDelimiter\brackets{\left\[}{\right\]}
%\DeclarePairedDelimiter\braces{\left\{}{\right\}}

\newcommand{\sheetnr}{1}
\newcommand{\issued}{23.10.2016}
\newcommand{\dueUni}{30.10.2016 10:00}
%\newcommand{\version}{1}   % if you need to release a corrected version, uncomment and increase this counter

\newcommand{\dd}{\text{d}}

\def\bra#1{\mathinner{\langle{#1}|}}
\def\ket#1{\mathinner{|{#1}\rangle}}
\def\braket#1{\mathinner{\langle{#1}\rangle}}
\def\Bra#1{\left<#1\right|}
\def\Ket#1{\left|#1\right>}
\def\delslash{\partial\hskip-1.1ex/}

\newcommand{\ee}{\ensuremath{{e^{+}e^{-}}}\xspace}
\newcommand{\mm}{\ensuremath{{\mu^{+}\mu^{-}}}\xspace}
\newcommand{\tautau}{\ensuremath{{\tau^{+}\tau^{-}}}\xspace}
\newcommand{\nunu}{\ensuremath{{\nu\nubar}}\xspace}
\newcommand{\ellell}{\ensuremath{{\ell^{+}\ell^{-}}}\xspace}
\newcommand{\ff}{\ensuremath{{f\hskip-0.3ex\bar{f}}}\xspace}
\newcommand{\ffbar}{\ff}
\newcommand{\Afb}{\ensuremath{A_\mathrm{FB}}\xspace}
\newcommand{\Alr}{\ensuremath{A_\mathrm{LR}}\xspace}
\newcommand{\Ae}{\ensuremath{\mathcal{A}_e}\xspace}
\newcommand{\Af}{\ensuremath{\mathcal{A}_f}\xspace}
\newcommand{\Pe}{\ensuremath{\mathcal{P}_e}\xspace}
\newcommand{\Pf}{\ensuremath{\mathcal{P}_f}\xspace}


%\showsolutions

\startsheet

Exercises should preferably be completed in English, but can also be done in German. 
Please write down what physics principles you start from and how you arrive at the solution.
A minimum of 60\% of the homeworks need to be handled for admission in the final exam. 

\exercise[3]{S.I units converstion}

Convert the following quantities into S.I. units. Note that $1~\mathrm{b} = 10^{-24}~\mathrm{cm}^{2}$.

\begin{subtasks}
\item 125 GeV into joules
\item 125 GeV into kilograms
\item 15 p$b$ into $meters^{2}$
\item 20 $TeV^{-1}$ into meters
\end{subtasks}

\exercise[8]{Special relativity}

\begin{subtasks}
\item Consider a symmetric 45 GeV $e^{+}e^{-}$ collider. If it was a fixed target experiment (the positron was stationary), calculate the energy needed for the incident electron.
\item Now consider an asymmetric $e^{+}e^{-}$ collider, where the electron has an energy of 9\,GeV and the positron has an energy of 3\,GeV. If the collision creates a particle of mass 10GeV, what speed does it travel at in the lab frame?
\end{subtasks}

\exercise[4]{Properties of operators} 

In this exercise some basic properties of operators are briefly reviewed. 

\begin{subtasks}
 \item Prove the following properties: 
 \begin{enumerate}
  %
   \item[i.] $[A,\,B] = -[B,\,A]$
  %
   \item[ii.] $[A,\,BC] = [A,\,B]C + B[A,\,C]$
  %
   \item[iii.] $[A,\,[B,\,C]] + [B,\,[C,\,A]] + [C,\,[A,\,B]] = 0$ 
  %
 \end{enumerate}
 %
  \item Consider the operators $\rm{X}$ and its corresponding derivate $D_{x}$ defined as 
  \begin{equation}
   \begin{aligned}
    {\rm{X\psi}}(x,y,z) & = x\psi(x,y,z)\,, \nonumber \\ 
    D_{x}\psi(x,y,z) & =\frac{\partial \psi(x,y,z)}{\partial x}\,. \nonumber
   \end{aligned}
  \end{equation}
    
  Calculate the commutator $[{\rm{X}},\,D_{x}]$
%
 \item The trace of an operator A on an N dimensional Hilbert space is defined as 
  \begin{equation} 
    {\rm{Tr}}(A) = \sum^{N}_{i=1}\bra{\psi_{i}} A \ket{\psi_{i}} \nonumber
  \end{equation}
  for any orthonormal set of basis vectors $\{\ket{\psi_{i}}\}$.
  Prove that ${\rm{Tr}}(AB) = {\rm{Tr}}(BA)$.  
%
\end{subtasks}

\exercise[5]{Potential barrier} 
  
  Consider an incident particle coming from $x = - \infty$ in a potential 

  \[ V(x) =  
  \left \{
  \begin{tabular}{cc}
    $~~~\,0$ , & $x < 0$     \\
    $+V_{0}$ , & $0 < x < a$ \\
    $~~~\,0$ , & $x > a$ \\ 
  \end{tabular}
  \right.
  \]
  Solve the stationary state with $E > V_{0}$ and calculate the reflection (R) and transmission (T) coefficients. 
  Plot the variation of T as a function of the width of the potential barrier for fixed values of $E$ and $V_{0}$. 
  Discuss the interpretation of the maximum of $T$.  

\end{document}
