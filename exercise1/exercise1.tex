\documentclass[11pt,a4paper]{article}

% Style
\usepackage{blattstil}
\usepackage{amsmath}
\usepackage{xspace}
\usepackage{hyperref}
\usepackage{amsthm, amssymb}

% for numbering give last exercise's number
\setcounter{aufgabennummer}{0}

\def\bra#1{\mathinner{\langle{#1}|}}
\def\ket#1{\mathinner{|{#1}\rangle}}
\def\braket#1{\mathinner{\langle{#1}\rangle}}
\def\Bra#1{\left<#1\right|}
\def\Ket#1{\left|#1\right>}
\def\delslash{\partial\hskip-1.1ex/}

\newcommand{\ee}{\ensuremath{{e^{+}e^{-}}}\xspace}
\newcommand{\mm}{\ensuremath{{\mu^{+}\mu^{-}}}\xspace}
\newcommand{\tautau}{\ensuremath{{\tau^{+}\tau^{-}}}\xspace}
\newcommand{\nunu}{\ensuremath{{\nu\nubar}}\xspace}
\newcommand{\ellell}{\ensuremath{{\ell^{+}\ell^{-}}}\xspace}
\newcommand{\ff}{\ensuremath{{f\hskip-0.3ex\bar{f}}}\xspace}
\newcommand{\ffbar}{\ff}
\newcommand{\Afb}{\ensuremath{A_\mathrm{FB}}\xspace}
\newcommand{\Alr}{\ensuremath{A_\mathrm{LR}}\xspace}
\newcommand{\Ae}{\ensuremath{\mathcal{A}_e}\xspace}
\newcommand{\Af}{\ensuremath{\mathcal{A}_f}\xspace}
\newcommand{\Pe}{\ensuremath{\mathcal{P}_e}\xspace}
\newcommand{\Pf}{\ensuremath{\mathcal{P}_f}\xspace}

\begin{document}

% Kopf des Blatts: lfd. Nummer und Abgabemodalitaeten
\kopf{1}{23.10.2015}{To be handed in during the lecture on 30.10.2016}

% Problem sets
\vspace{-5mm}

Exercises should preferably be completed in English, but can also be done in German. Ansatz and solution are graded. Therefore, write down what physics principles you start from and how you arrive at the solution.

\begin{aufgabe}[3]

Convert the following quantities into S.I. units. Note that $1~\mathrm{b} = 10^{-24}~\mathrm{cm}^{2}$.

\begin{enumerate}
\item 125 GeV into joules
\item 125 GeV into kilograms
\item 15 p$b$ into $meters^{2}$
\item 20 $TeV^{-1}$ into meters
\end{enumerate}

\end{aufgabe}

\begin{aufgabe}[10]

\begin{enumerate}
\item Consider a symmetric 45 GeV $e^{+}e^{-}$ collider. If it was a fixed target experiment (the positron was stationary), calculate the energy needed for the incident electron.
\item Now consider an asymmetric $e^{+}e^{-}$ collider, where the electron has an energy of 9\,GeV and the positron has an energy of 3\,GeV. If the collision creates a particle of mass 10GeV, what speed does it travel at in the lab frame?
\end{enumerate}

\end{aufgabe}

\begin{aufgabe}[3]

Do some quantum mechanics.

\end{aufgabe}

\end{document}
