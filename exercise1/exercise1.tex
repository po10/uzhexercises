\documentclass[11pt,a4paper]{article}

% Style
\usepackage{blattstil}
\usepackage{amsmath}
\usepackage{xspace}
\usepackage{hyperref}
\usepackage{amsthm, amssymb}

% for numbering give last exercise's number
\setcounter{aufgabennummer}{0}

\def\bra#1{\mathinner{\langle{#1}|}}
\def\ket#1{\mathinner{|{#1}\rangle}}
\def\braket#1{\mathinner{\langle{#1}\rangle}}
\def\Bra#1{\left<#1\right|}
\def\Ket#1{\left|#1\right>}
\def\delslash{\partial\hskip-1.1ex/}

\newcommand{\ee}{\ensuremath{{e^{+}e^{-}}}\xspace}
\newcommand{\mm}{\ensuremath{{\mu^{+}\mu^{-}}}\xspace}
\newcommand{\tautau}{\ensuremath{{\tau^{+}\tau^{-}}}\xspace}
\newcommand{\nunu}{\ensuremath{{\nu\nubar}}\xspace}
\newcommand{\ellell}{\ensuremath{{\ell^{+}\ell^{-}}}\xspace}
\newcommand{\ff}{\ensuremath{{f\hskip-0.3ex\bar{f}}}\xspace}
\newcommand{\ffbar}{\ff}
\newcommand{\Afb}{\ensuremath{A_\mathrm{FB}}\xspace}
\newcommand{\Alr}{\ensuremath{A_\mathrm{LR}}\xspace}
\newcommand{\Ae}{\ensuremath{\mathcal{A}_e}\xspace}
\newcommand{\Af}{\ensuremath{\mathcal{A}_f}\xspace}
\newcommand{\Pe}{\ensuremath{\mathcal{P}_e}\xspace}
\newcommand{\Pf}{\ensuremath{\mathcal{P}_f}\xspace}

\begin{document}

% Kopf des Blatts: lfd. Nummer und Abgabemodalitaeten
\kopf{1}{23.10.2015}{To be handed in during the lecture on 30.10.2016}

% Problem sets
\vspace{-5mm}

Exercises should preferably be completed in English, but can also be done in German. Ansatz and solution are graded. Therefore, write down what physics principles you start from and how you arrive at the solution.

\begin{aufgabe}[3]

Convert the following quantities to units of energy [eV] (in natural units $\hbar = c = 1$). Mind that $1~\mathrm{b} = 10^{-24}~\mathrm{cm}^{2}$.

\begin{enumerate}
\item Something 
\end{enumerate}

\end{aufgabe}

\begin{aufgabe}[10]

Consider an $e^{+}e^{-}$ collider which is designed to work at the Z boson mass threshold (90GeV). Work out the cms energy if it was a fixed target experiment.

\begin{enumerate}
\item Do some stuff perhaps

\end{enumerate}

\end{aufgabe}

\begin{aufgabe}[3]

Do some quantum mechanics.

\end{aufgabe}

\end{document}
